%-------------------------------------------------------------------------------
%	SECTION TITLE
%-------------------------------------------------------------------------------
\cvsection{Research Articles}


%-------------------------------------------------------------------------------
%	CONTENT
%-------------------------------------------------------------------------------
\begin{cventries}

%---------------------------------------------------------
  \cventry
    {Preprint(s)} % Role
    {} % Title
    {} % Location
    {} % Date(s)
    {
      \begin{cvitems} % Description(s)
      	\item{\textbf{A. Panigrahi}, R. Moessner, B. Roy; \textit{Non-Hermitian dislocation modes: Stability and melting across exceptional points} (2021) \boxed{\text{\href{https://arxiv.org/abs/2105.05244}{arXiv:2105.05244}}}}
      	\item{\textbf{A. Panigrahi}, S. Mukerjee; \textit{Energy magnetization and transport in systems with a non-zero Berry curvature in a magnetic field} (2021)
      	\boxed{\text{\href{https://arxiv.org/abs/2111.08026}{arXiv:2111.08026}}}}
        \item{\textbf{A. Panigrahi}, V. Juri\v{c}i\'c, B. Roy; \textit{Projected Topological Branes} (2021)}\boxed{\text{\href{}{arXiv:}}}
      	%\item{\href{https://github.com/apandada1/apandada1.github.io/raw/master/articles/nanoheatengines.pdf}{A Study of Generation of Classical Squeezed States Using Stochastic Force, and their Applications in Building Highly Efficient Heat Engines} (2019)}
        %\item {Review article - \href{https://github.com/apandada1/apandada1.github.io/raw/master/articles/tachyon.pdf}{A detailed example of how causality is violated when information travels faster than speed of light in vaccum} (2018)}
        %\item{Review article - \href{https://github.com/apandada1/apandada1.github.io/raw/master/articles/doppler_effect.pdf}{Doppler effect of electromagnetic waves in refractive medium} (2018)}
        %\item{\href{https://github.com/apandada1/apandada1.github.io/raw/master/articles/Harmonic_Mean.pdf}{A Geometric Method to obtain Harmonic Mean of Two numbers} (2016)}
      \end{cvitems}
    }
%---------------------------------------------------------
\end{cventries}
