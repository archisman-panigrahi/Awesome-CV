%-------------------------------------------------------------------------------
%	SECTION TITLE
%-------------------------------------------------------------------------------
\cvsection{Achievements}


%-------------------------------------------------------------------------------
%	SUBSECTION TITLE
%-------------------------------------------------------------------------------



%-------------------------------------------------------------------------------
%	CONTENT
%-------------------------------------------------------------------------------
\begin{cvhonors}

  \cvhonor
{C.G.P.A} % Award
{9.8/10 in B.S. (Research), highest GPA in batch} % Event
{IISc, Bangalore} % Location
{2017-21} % Date(s)
%---------------------------------------------------------
  \cvhonor
    {1st rank (99.2 \%) in Board} % Award
    {in Higher Secondary Examination} % Event
    {West Bengal, India} % Location
    {2017} % Date(s)

%---------------------------------------------------------
  \cvhonor
    {} % Award
    {10th rank in National Entrance Screening Test (NEST)} % Event
    {India} % Location
    {2017} % Date(s)

%---------------------------------------------------------
  \cvhonor
	{} % Award
	{Qualified for JEE Mains (All India Rank - 381) - an all India Engineering entrance} % Event
	{} % Location
	{2017} % Date(s)
	 
	\cvhonor
	{} % Award
	{Qualified for JEE Advanced examination (All India Rank- 543), Entrance examination of Indian Institute(s) of Technology (IIT)} % Event
	{} % Location
	{2017} % Date(s)
  \cvhonor
    {} % Award
    {Qualified for Indian Statistical Institute, Kolkata and Chennai Mathematical Institute} % Event
    {} % Location
    {2017} % Date(s)

%---------------------------------------------------------
  \cvhonor
    {} % Award
    {Qualified for K.V.P.Y (All India Rank - 128)} % Event
    {} % Location
    {2015} % Date(s)

%---------------------------------------------------------
  \cvhonor
    {2nd rank (97.57 \%) in Board} % Award
    {in Secondary Examination} % Event
    {West Bengal, India} % Location
    {2015} % Date(s)
    \\\\

%---------------------------------------------------------
\end{cvhonors}

